\documentclass[12pt]{amsart}

\usepackage{verbatim}

% math macros
\newcommand\bb{\mathbf{b}}
\newcommand\bbf{\mathbf{f}}
\newcommand\bn{\mathbf{n}}
\newcommand\bq{\mathbf{q}}
\newcommand\bu{\mathbf{u}}
\newcommand\bv{\mathbf{v}}
\newcommand\by{\mathbf{y}}

\newcommand\bQ{\mathbf{Q}}
\newcommand\bV{\mathbf{V}}
\newcommand\bX{\mathbf{X}}

\newcommand\CC{\mathbb{C}}
\newcommand{\DDt}[1]{\ensuremath{\frac{d #1}{d t}}}
\newcommand{\ddt}[1]{\ensuremath{\frac{\partial #1}{\partial t}}}
\newcommand{\ddx}[1]{\ensuremath{\frac{\partial #1}{\partial x}}}
\newcommand{\ddy}[1]{\ensuremath{\frac{\partial #1}{\partial y}}}
\newcommand{\ddxp}[1]{\ensuremath{\frac{\partial #1}{\partial x'}}}
\newcommand{\ddz}[1]{\ensuremath{\frac{\partial #1}{\partial z}}}
\newcommand{\ddxx}[1]{\ensuremath{\frac{\partial^2 #1}{\partial x^2}}}
\newcommand{\ddyy}[1]{\ensuremath{\frac{\partial^2 #1}{\partial y^2}}}
\newcommand{\ddxy}[1]{\ensuremath{\frac{\partial^2 #1}{\partial x \partial y}}}
\newcommand{\ddzz}[1]{\ensuremath{\frac{\partial^2 #1}{\partial z^2}}}
\newcommand{\Div}{\nabla\cdot}
\newcommand\eps{\epsilon}
\newcommand{\grad}{\nabla}
\newcommand{\ihat}{\mathbf{i}}
\newcommand{\ip}[2]{\ensuremath{\left<#1,#2\right>}}
\newcommand{\jhat}{\mathbf{j}}
\newcommand{\khat}{\mathbf{k}}
\newcommand{\nhat}{\mathbf{n}}
\newcommand\lam{\lambda}
\newcommand\lap{\triangle}
\newcommand\Matlab{\textsc{Matlab}\xspace}
\newcommand\RR{\mathbb{R}}
\newcommand\vf{\varphi}


\title{A different approach to minimal surfaces: \\ Documentation of code \texttt{blow.py}}

\author{Ed Bueler}


\begin{document}

\begin{abstract}
This brief document explains the new code \texttt{blow.py}.  Note that \texttt{minimal.py} solves the minimal surface equation for a surface of the form $z=u(x,y)$.  The new code instead solves for a parameterized surface of the form $\bX(s,t) = (x(s,t),y(s,t),z(s,t))$.  In this paper we derive a (not apparently well-known) first-variation formula for the Dirichlet problem of finding $\bX: \Omega \subset \RR^2 \to \RR^3$ such that $\bX$ is given along $\partial \Omega$.  The weak formulation corresponding to this first-variation formula is not well-posed, as one can see by its invariance with respect to reparameterization.  However, it apparently becomes well-posed if we add a Laplacian term for two of the components.  [FIXME: Maybe this does relate to isothermal coordinates after all.]
\end{abstract}

\maketitle

\thispagestyle{empty}

%\baselineskip=12pt
%\parskip=5pt

\section{Introduction}

FIXME

%         References
%\bibliography{XX}
%\bibliographystyle{siam}

\end{document}
